% +- 1 página
% Historico do time e do nome
% Objetivos
% Motivaçao
%  - Instrumento de avaliaçao da dissertacao
%  - Ganhar experiencia no desenvolvimento de sistemas multiagentes.

\section{Introduction}

% \begin{enumerate}
% \item What was the motivation to participate in the contest?
% \item What is the (brief) history of the team? (MAS course project,  thesis evaluation, $\ldots$)
% \item What is the name of your team?
% \item How many developers and designers did you have? At what level of education are your team members?
% \item From which field of research do you come from? Which work is related?
% \end{enumerate}

The empirical evaluation of proposals in the context of Multi-Agent Systems (MAS) is a quite complex task and the Multi-Agent Programming Contest \cite{behrens:2010,behrens:2011b}\footnote{\url{http://multiagentcontest.org}} offers an useful context for doing this evaluation. In particular, the latest Mars scenario has emphasised solutions based on cooperation, coordination, and decentralisation which are important topics for our research. This contest is thus selected as the environment to evaluate the proposals being developed by the authors in their master and Phd thesis. Among the authors, we have one PhD student, three master students, and one undergraduate student. The main approach is ($i$) to develop a \emph{base} MAS for the contest, then ($ii$) the master and PhD students will change the base system using their corresponding proposals, and finally ($iii$) each proposal can be evaluated and compared against the base system. In this paper we report the development and the main features of this base team, called SMADAS (the acronym of our research group). Another objective for attending the contest is to improve the experience in developing MAS. Since most of the authors are just beginning on the domain, the concrete experience is important for their overall learning and maturity in critical analysis. 


