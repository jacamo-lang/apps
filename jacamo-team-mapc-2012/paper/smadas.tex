\documentclass{llncs}
\usepackage{todonotes}
\usepackage{url}
\usepackage{booktabs}
\usepackage{subfig} % este é para fazer sub figuras

%\usepackage{float}
%\floatstyle{boxed} 
%\restylefloat{figure}

\begin{document}
\title{SMADAS: a Cooperative Team for the Multi-Agent Programming Contest using Jason}
%\author{[Authors]}
%\institute{[Affiliation]}
\author{	Maicon Rafael Zatelli \and 
		Daniela Maria Uez \and
		Jos\'{e} Rodrigo Neri \and \\
		Tiago Luiz Schmitz \and
		J\'{e}ssica Pauli de Castro Bonson \and 
		Jomi Fred H\"{u}bner }
		

\institute{Department of Automation and Systems Engineering \\ Federal University of Santa Catarina \\ CP 476, 88040-900 Florian\'{o}polis - SC - Brasil \\ 
\email{	{\{xsplyter,dani.uez,jrf.neri,tiagolschmitz,jpbonson\}@gmail.com,
		jomi@das.ufsc.br}}}

\maketitle
\begin{abstract} 
In this paper we describe the SMADAS system used for the Multi-Agent Programming Contest in 2012. This contest offers an useful context to evaluate tools, techniques, and languages for programming MAS. It is also a good opportunity to learn agent programming and test new features we are developing in our projects. Throughout the paper we highlight the main strategies of our team and comment on the advantages and disadvantages of our system as well as some improvements that still could be done. One important result from this experience regards the agent programming language we used, it provides suitable abstractions for the development of complex system and shows an increment in its maturity since no bugs was discovered this year.
% We finished the paper by pointing some directions and issues to regard for the future tournaments such as the use of organization and interaction as first class abstractions.
\end{abstract}

% -------------------------------
% Inclusao dos capitulos do texto
% +- 1 página
% Historico do time e do nome
% Objetivos
% Motivaçao
%  - Instrumento de avaliaçao da dissertacao
%  - Ganhar experiencia no desenvolvimento de sistemas multiagentes.

\section{Introduction}

% \begin{enumerate}
% \item What was the motivation to participate in the contest?
% \item What is the (brief) history of the team? (MAS course project,  thesis evaluation, $\ldots$)
% \item What is the name of your team?
% \item How many developers and designers did you have? At what level of education are your team members?
% \item From which field of research do you come from? Which work is related?
% \end{enumerate}

The empirical evaluation of proposals in the context of Multi-Agent Systems (MAS) is a quite complex task and the Multi-Agent Programming Contest \cite{behrens:2010,behrens:2011b}\footnote{\url{http://multiagentcontest.org}} offers an useful context for doing this evaluation. In particular, the latest Mars scenario has emphasised solutions based on cooperation, coordination, and decentralisation which are important topics for our research. This contest is thus selected as the environment to evaluate the proposals being developed by the authors in their master and Phd thesis. Among the authors, we have one PhD student, three master students, and one undergraduate student. The main approach is ($i$) to develop a \emph{base} MAS for the contest, then ($ii$) the master and PhD students will change the base system using their corresponding proposals, and finally ($iii$) each proposal can be evaluated and compared against the base system. In this paper we report the development and the main features of this base team, called SMADAS (the acronym of our research group). Another objective for attending the contest is to improve the experience in developing MAS. Since most of the authors are just beginning on the domain, the concrete experience is important for their overall learning and maturity in critical analysis. 


 	% Jomi
\section{System Analysis and Design}
\label{analysis}

For this year's contest, we opted for developing a new team using the \jacamo~\cite{boissier:2011} platform instead of just improving the last year team. The programming model of \jacamo\ provides high-level support for developing MAS considering agents, environment, and organization as \emph{first-class} entities. The development of these three dimensions is based on three different technologies: Jason \cite{bordini:2007}, for programming the agents; \cartago \cite{ricci:2011}, for programming the environment; and \moise \cite{hubner:2007} for programming the organization. Thus, this year the organization and environment, which were previously implemented as part of the agent code, have now been programmed with proper organizational and environmental elements according to the aforementioned models and technologies.

\subsection{Organizational Dimension}

As \jacamo supports organizational programming, part of the coordination of team agents is modeled in the organizational dimension instead of being modeled as skills of the agents. The organization provides guidelines for the achievement of the overall system goals, but the agents remain autonomous to decide how to achieve them. For example, the organization informs that an agent is obligated to probe the vertices, but the agent is autonomous about ``how'' to do it, based on its local knowledge about the world. However, the autonomy can be constrained by means of organizational norms.

Fig.~\ref{fig:org_ss} shows the structural specification (SS) of the team using the \moise\ notation. Notice that the SS is designed based on the roles of the contest scenario. The team is divided into two \emph{sub-teams}. Besides, the team has three minor subgroups: \emph{special operations}, \emph{special exploration}, and \emph{pivots}. An agent can play more than one role at the same time. For example, an \emph{explorer} can also play \emph{explorer leader} and \emph{special explorer} roles. One agent plays the role \emph{leader} and is responsible to manage the overall organization (e.g. designating the roles of the other agents). The functional specification (FS) (Fig.~\ref{fig:org_fs})) specifies the goals (i.e. specific states of affairs) that the agents must achieve and distributes these goals to the agents (by means of missions). The overall goal (\emph{domain mars}) consists in a decomposition tree where the leaves are the goals that can be achieved by the agents. The goals are grouped in four \emph{missions} ($m1$, $m2$, $m3$, and $m4$). Finally, the normative specification (NS) (Fig.~\ref{fig:org_ns}) relates roles to missions. For example, as the \emph{explorer leader} is obligated to commit to mission $m4$, it is obligated to achieve the goals \emph{define initial hill}, \emph{conclude first phase}, and \emph{dismiss agent}.

\begin{figure}[th]
 \vspace{-5mm}
 \centering
 \subfigure[Structural specification]{\includegraphics[width=0.54\textwidth]{figs/ss.pdf}\label{fig:org_ss}}
 \subfigure[Functional specification]{\includegraphics[width=0.45\textwidth]{figs/fs.pdf}\label{fig:org_fs}}
 \subfigure[Normative specification]{\setlength{\tabcolsep}{12pt}
\begin{tabular}{llll}
\emph{norm} & \emph{role} &   & \emph{mission}\\
\midrule
n1 & explorer & obligation & m1\\
n2 & sentinelLeader & obligation & m2\\
n3 & inspector & obligation & m3\\
n4 & explorerLeader & obligation & m4
\end{tabular}

\label{fig:org_ns}}
 \vspace{-3mm}\caption{Organizational Specification}
 \label{fig:org}
 \vspace{-3mm}
\end{figure}

Along the development of the team, we performed some experiments introducing \emph{count-as rules}~\cite{brito:2012}. Count-as rules changes in the organization as the result of facts occurring in the environment. For example, without count-as rules, a specific agent has to set up the organizational infrastructure and explicitly adopt the role of \emph{leader}. With count-as rules, it is possible to state that such setup \emph{counts-as} the adoption of the \emph{leader} role. In another example, without count-as rules, the agents have to reason about the organizational structure, checking their roles, and committing to missions according to that roles. With count-as rules, it is possible to model that playing of a specific role \emph{counts-as} the commitment to a specific mission. The use of count-as rules simplifies the reasoning and action of the agents, as they do not need to perform some actions on the organization (e.g. they do not need to reason and to act to commit to missions). Besides, count-as rules contribute to keep the organization in a consistent state as some organizational actions do not depend on the agents actions. Due to time constraints, the count-as rules were not added to the tournament team. However, the experiments indicate that the rules seem a suitable approach for further versions of the team.

\subsection{Environment Dimension}\label{sec:env}

The environment for our agents has two parts. The first part provides integration with the contest simulation by means of EISMASSim framework~\cite{behrens:2011} and is well defined in the contest documentation. The second part is provided by \jacamo\ artifacts that agents perceive and use to achieve their goals. In our team, the information about the inspected enemies is managed by an artifact.  We also conceived an artifact responsible to manage all the graph structure. However, since we used the same graph structure and algorithms of the last year, which were based on Java shared memory, we decided to not move this previous implementation into a specific \cartago\ artifact because it would require more time. Therefore, part of the environment is developed in \cartago\ and another part was kept in ``pure Java'', accessed by means of Jason internal actions, as we did last year. It is a future work to unify all perceptions and actions under the \cartago\ approach.


\subsection{Agent Dimension}

The agents may behave proactively or reactively, in accordance with the needs. For example, a damaged agent will proactively call a repairer and all agents promptly react to the environment events, like the start of a new  simulation step. 

Agents share information by two mechanisms: messages and blackboards. Since it was not so appropriate to broadcast everything between the agents because we would have $28 \times 27$ messages\footnote{The team is composed of 28 agents.}, we chose to send messages about few things. For example, when agents are disabled, they call a repairer; when enemies invade some team area, the saboteurs are notified; and when some vertex is probed, the explorers broadcast the value of the vertex. Some information is also important to exchange between agents of the same kind to avoid them performing the same action. For example, when there are two saboteurs with enemies in the same vertex, they need to communicate to decide which enemy each one will attack. We used the same solution for explorers and repairers to avoid repairing the same agent and to avoid probing the same vertex. Messages are also used to inform agents about the zones they should help to conquer.

The second (indirect) communication mechanism is the use of blackboards and artifacts as commented on Sec.~\ref{sec:env}. In this case, the agents share the graph structure, the information about the inspected agents, and the position of the enemies and team mates. Finally, the remaining data, such as their own health, energy, zone scores, or visible vertices, is private for each agent.

We defined a priority among the agents to avoid conflicting actions (like two agents probing the same vertice). The agent with the highest priority chooses its action first and informs the others of the same role about its decision. Then the agent with the second highest priority does the same and so forth.\footnote{Notice that sharing the information about the chosen action is not enough to solve the problem. Some coordination is required to efficiently solve it. Although this coordination is an organizational issue, this priority solution was coded in the agent dimension and it remains as a future work to model it in the organizational dimension.} However, actions like survey and inspect do not follow this priority approach. That means two agents can inspect or survey at the same target to try to guarantee some of them will be successful. 
%TODO: For instance, an agent chooses the action ``a'', another agent chooses the same, they then share these decisions, discover the conflict, both change their decision possibly for another identical action, announce, ...   eu tirarei essa parte...
\subsection{Testing}

To develop the team we used a particular incremental process. We performed weekly meetings to define the team strategies. These strategies were implemented and tested during the week and, in the next meeting, these results were considered to define new improvements. To evaluate the team strategies and ensure the competitiveness, we tested our team by simulating a great amount (more than 1000) of matches against our 2012 team~\cite{smadas:2012}, the 2012 Python-DTU team, and previous versions of our current team. The aim of the tests was to evaluate the overall performance of the team in different maps, adopting different strategies and facing different strategies from the opponent. In addition, we participated in all test matches during the testing phase to evaluate the connection and a couple of strategies.
		% Dani
\section{Strategies, Details and Statistics}
\label{secStrategies}

In this section, we describe the main strategies of our team (Sec.~\ref{sec:teamStrategies}) and we highlight the main results that we got (Sec.~\ref{sec:comparisonOtherTeams}).

\subsection{Team Strategies}
\label{sec:teamStrategies}

The strategy of the team has two moments. In the former, the agents explore the map to obtain achievement points and to define good zones as soon as possible. Thus, it is possible to get a good score in the first steps. In the latter, the agents start to conquer and protect several small zones. During the whole match, the saboteurs disturb the enemy and the repairers help disabled agents.


\begin{figure}[th]
 \vspace{-5mm}
 \centering
 \subfigure[Hill]{\includegraphics[width=0.4\textwidth]{figs/hill.png}\label{fig:hill}}
 \subfigure[Pivots]{\includegraphics[width=0.3\textwidth]{figs/pivots.png}\label{fig:pivots}}
 \subfigure[Islands]{\includegraphics[width=0.17\textwidth]{figs/islands.png}\label{fig:islands}}
 \vspace{-3mm}\caption{Hills, pivots, and islands}
 \label{fig:hillislandpivot}
 \vspace{-3mm}
\end{figure}

The good zones are defined in terms of \emph{hills}, \emph{pivots}, and \emph{islands}. A hill (the big zone in Fig.~\ref{fig:hill}) is a zone formed by several vertices that have a good value and are in the same region of the map. As in the 2012 team, the agents try to discover two hills. The hills are defined as follows: for each vertex $v$ of the graph, the algorithm sums the values of all vertices up to two hops away from $v$, including $v$. The two vertices with the highest sums are defined as the center of the hills, and then the agents try to stay on the neighborhoods. The agents control the hills simply moving to the border of the hills in order to expand them. If they break the zone, they come back to the previous places and try to expand again. We also defined that the sentinels need to conquer the best vertices in the hills and stay over them all the time until the strategy changes. Sometimes, it induces the opponents to avoid those places and we guarantee a fixed gain of scores of the hills, even if the enemy is disturbing. Moreover, the explorers of the group \emph{special exploration} prefer to probe first the vertices in the hills, because it increases the gain of points in the first steps.


\begin{figure}[th]
 \vspace{-5mm}
 \centering
 \subfigure[]{\includegraphics[width=0.15\textwidth, height=0.15\textwidth]{figs/protectIsland1.png}\label{fig:protectIsland1}}
 \subfigure[]{\includegraphics[width=0.15\textwidth, height=0.15\textwidth]{figs/protectIsland2.png}\label{fig:protectIsland2}}
 \subfigure[]{\includegraphics[width=0.15\textwidth, height=0.15\textwidth]{figs/protectIsland3.png}\label{fig:protectIsland3}}
 \subfigure[]{\includegraphics[width=0.15\textwidth, height=0.15\textwidth]{figs/protectIsland4.png}\label{fig:protectIsland4}}
 
 \vspace{-3mm}
 
 \subfigure[]{\includegraphics[width=0.65\textwidth]{figs/protectIsland5.png}\label{fig:protectIsland5}}
 \vspace{-3mm}\caption{Protecting islands}
 \label{fig:protectIslands}
 \vspace{-3mm}
\end{figure}

Islands (Fig.~\ref{fig:islands}) are regions of the map that can be conquered by a single agent. An island is a zone that has only one vertex (a \emph{cut vertex}) in common with the remaining graph. They are found by disconnecting the edges of each cut vertex of the graph. It produces two disconnected subgraphs, and the smallest one, plus the cut vertex, are an island. If there are enemies on the island, the controller agent will go to the same vertex of the enemy. Thus, both teams do not get the points of that island. The figures~\ref{fig:protectIsland1}, ~\ref{fig:protectIsland2},~\ref{fig:protectIsland3}, and~\ref{fig:protectIsland4} illustrate this situation. In addition, the controller agent notifies the saboteur leader about the invader. If the saboteur leader is already busy protecting another island, the saboteur leader calls the saboteur helper of the group \emph{special operations}. If both are busy the saboteur leader keeps a list of islands with enemies for a further attack. Fig.~\ref{fig:protectIsland5} illustrates a probable call of the saboteur leader (the diamond not at the cut vertex) that is going to fight against an explorer (the circle) and a saboteur (the diamond at the cut vertex).

Pivots (Fig.~\ref{fig:pivots}) are regions of the map that can be conquered by just two agents. For each pair of vertices ($u$,$v$) we search all vertices $w$ connected to $u$ and $v$. For all vertices $w$ (including $u$ and $v$) we also search all vertices only connected to these vertices. For example, if there is a vertice $k$ connected only to the vertice $w$, then $k$ also belongs to the pivot. Furthermore, if there is an island connected to some of these vertices we consider all the vertices of the island. The best pivots are chosen considering the sum of all vertices. The agents that control pivots do not move away from their places, since most of the time the enemy does not stay fixed in those places. However, if the enemy stays in the same vertex both teams do not get the points, and so our team also cancels the enemy strategy. This is another reason to not leave the vertices. Otherwise the opponent will get the points.

The hills are defined in the first phase of our strategy, until around step 130. We chose to use hills instead of islands and pivots in the beginning of the match because most of the vertices are still unprobed and so we would not get so many points. The use of hills can keep all agents together and getting higher points because the zones are bigger. After a while the agents start conquering pivots and islands. The agents also need to decide if it is better to conquer two islands or one pivot. The decision is taken by simply summing the value of two islands and comparing with the pivots. If the two islands provide the same gain of the pivot or if they are more valuable, the agents will prefer to conquer two islands instead of conquering one pivot. The pivots and islands are very stable and so our agents are not disturbed by the enemy while our agents can disturb their zones since it is harder to protect a big zone than several small zones. 

\begin{table}[th]
\vspace{-5mm}
\begin{center}

	\begin{tabular}{l c c c c c}
		\toprule
		Action & Repairer & Saboteur & Explorer & Sentinel & Inspector \\
		\midrule
		%\midrule
		attack &  & x &  &  &  \\
		%\midrule
		repair & x &  &  &  &  \\
		%\midrule
		parry & x &  &  & x &  \\
		%\midrule
		probe &  &  & x &  &  \\
		%\midrule
		inspect &  &  &  &  & x \\
		%\midrule
		recharge & x & x & x & x & x \\
		%\midrule
		goto & x & x & x & x & x \\
		%\midrule
		survey & x & x & x & x & x \\
		\bottomrule            
                
	\end{tabular}
\end{center}
\caption{Implemented strategies by agent type. \label{tab:tabStrategies}}
\vspace{-5mm}
\end{table}

The achievements continue to be as important as in the last year. We decided do not waste money buying items and get as much achievements as possible and as soon as possible, since they accumulate in each step. We made this decision since in our tests we did not see any advantage in buying items. Finally, the specific strategies of each kind of agent are explained below while Table~\ref{tab:tabStrategies} summarizes the strategies implemented for each kind of agent.

\vspace{-3mm}

\begin{description}
 \item[Explorer:] the explorers have an important role in the beginning of the match. They need to probe all vertices as soon as possible. To do so, the explorers avoid performing the survey action and conquering zones until they have probed all vertices. Furthermore, the \emph{explorer leader} defines the initial two hills.
 \item[Saboteur:] the main aims of the saboteurs are to protect the islands and disturb the enemy. The saboteur with the role \emph{saboteur chaser} has the aim to attack mainly explorers, inspectors, sentinels, and repairers to avoid staying in the same vertex fighting against saboteurs all the time. The saboteurs with the roles \emph{saboteur leader} and \emph{saboteur helper} have the main aim to protect islands against enemies. The \emph{saboteur leader} is also the main contact of the other agents to ask for help. In other cases, the saboteurs simply search and destroy enemies. In addition, the saboteurs attack following a priority. First of all they prefer to attack saboteurs, followed by explorers, inspectors, repairers, and sentinels. The saboteurs prefer to attack explorers and inspectors because they can not parry.
 \item[Repairer:] the main aim of the repairers is to keep the agents enabled. All agents that are disabled inform the \emph{repairer leader}. The \emph{repairer leader} asks all the other repairers if they can help the disabled agents. If a repairer is not committed to an agent and it is not getting high points and it is enabled, then it is apt to help the other agent. All apt repairers inform the path size until the disabled agent and the \emph{repairer leader} chooses the closest repairer to help the disabled agent. To make the repair operation faster, both the repairer agent and the disabled agent follow the same path to meet each other. If there is some repairer next to the disabled agent, that repairer will repair the agent and the agent will cancel the appointment with its repairer. Finally, when the repairers are not helping the disabled agents, they also go to the pivots or islands that they belong to.
 \item[Sentinel:] the sentinels always protect the best zones, since they are harder to get disabled and they can parry. They are usually avoided by the enemy because they do not have an important role like repairers, explorers, and saboteurs. Therefore, the sentinels are not disturbed so often. The sentinels also try to survey if they find some vertex with edges with unknown value. Finally, the \emph{sentinel leader} has the aim to define the islands and pivots and inform the agents about it.
 \item[Inspector:] the inspectors protect the next best zones, since they are also not so disturbed by the enemies. Their aim is to inspect all the enemy agents and they do it just once, since we do not care about what the enemy is buying. Doing so, the inspectors can stay in the same vertices until the end of the match, getting more points than by inspecting enemies.
\end{description}

\subsection{Comparison to Other Teams}
\label{sec:comparisonOtherTeams}

\begin{figure}[th]
 \vspace{-5mm}
 \centering
 \subfigure[Scores]{\includegraphics[width=0.49\textwidth]{figs/Scores.png}\label{fig:Scores}}
 \subfigure[ZoneStabilities]{\includegraphics[width=0.49\textwidth]{figs/ZoneStabilities.png}\label{fig:ZoneStabilities}}
 
 \vspace{-3mm}
 
 \subfigure[AchievementPoints]{\includegraphics[width=0.49\textwidth]{figs/AchievementPoints.png}\label{fig:AchievementPoints}}
 \subfigure[ZonesScores]{\includegraphics[width=0.49\textwidth]{figs/ZonesScores.png}\label{fig:ZonesScores}} 
 \vspace{-3mm}\caption{Statistics}
 \label{fig:statistics}
 \vspace{-3mm}
\end{figure}

In order to highlight the main results of our strategy we chose to use some statistics of the second match against the GOAL-DTU team \footnote{\url{http://multiagentcontest.org/downloads/func-startdown/1716/}}, which got the second place. Due our strategy of small zones, we can verify in the Fig.~\ref{fig:ZoneStabilities}  that after step 325 our team (blue) kept almost the same zones until the end of the match. This behavior is usual in all matches against the other teams. The reason is that no matter what the opponent does, our agents will rarely leave their positions.

Another interesting result can be drawn from the zones scores plot (Fig.~\ref{fig:ZonesScores}). We can see that our team (blue) kept getting almost the same zone scores after step 350 and always more than the opponent. The phase of hills can also be noticed in the beginning of the match, between step 25 until around step 130, where the team is getting high scores because of the big zones. After step 130 our team started to conquer small zones (pivots and islands) and, therefore, spreading the agents out over the whole map. It is also possible to see that, sometimes, our zone scores were lower than the enemies. This was an expected behavior when the agents were still changing their positions because the explorers were still probing new vertices. We can see it after step 250 and after around step 315, where our team decreased the gain of zone scores. After probing all vertices (around step 325), the agents started to get higher scores because they defined the fixed zones and all agents were participating.

We can see the same behavior in Fig.~\ref{fig:Scores}, where the opponent score gets closer and then the difference of scores increases again. Notice that after step 325 the difference of scores increased continuously because all vertices were probed. On the other hand, in the Fig.~\ref{fig:AchievementPoints} we can see that our team always has more achievement points than the opponent after step 125. It means we are getting more points because the opponent was buying items while our team was saving money. 	% Maicon
\section{Software Architecture}

As we done in the edition of 2012~\cite{smadas:2012}, in the current edition of MAPC we used the EISMASSim~\cite{behrens:2011} to communicate with the contest server. However, while in the previous edition the team was developed essentially using Jason, in the current edition, our team has been developed with \jacamo\  platform. This was the main change in the software architecture for this year. Furthermore, even with the increase number of agents, from 20 to 28, we were still able to run the agents in a single machine, therefore we decided to avoid distributing the agents between several machines. We also made several contributions for the tools that we used in this year. In Jason, we added features to handle goals with deadlines, new mechanisms for the \texttt{.wait} internal action, and we fixed some bugs. In \moise, we added a new feature to reset organizational goals to avoid creating new schemes at runtime and we added an organizational monitor accessed via HTTP, so that we were able to watch our team organization remotely.

The source code of the team has 3794 lines of Jason code, 135 for \moise, 96 for \cartago, and 4434 for Java, totaling 8459 lines. Although the implemented strategies of these year are more complex, we can notice that the number of lines coded in Jason has decreased from 5504 in last year's team to 3794 this year. It is an expected consequence of the organization and the environment programming available in \jacamo. Coordination strategies that previously required several lines of Jason code, could now be coded in a few lines of \moise, since \moise\ is a proper language for that.  Not only have we reduced the size of the programs, but the new approach has allowed us to debug and change the organization of the team quite easily. Instead of monitoring the agents internal state, we can now monitor the state of the organization, which is a more general view of the state of the team. Since the organizational program is the same as the specification, to change the team sometimes is simply reduced to update the organization. For instance, to change the order of organizational goals, we simply need to change the scheme of Fig.~\ref{fig:org_fs}. 	% Tiago
% Não sei se fica esse título mesmo... +- 2 páginas
% Conteúdo
% - Estatísticas do time

% - Desempenho usando a técnica do Hulk e sem o Hulk - comparação entre nossos times e nosso time contra Python-DTU 
%Hulk: it is the agent responsible for buying;

% - Desempenho usando 1 morrinho ou 2 morrinhos: uso dessa técnica contra o time chinês



% Vantagens e desvantagens.

\section{Results}
%será que devia dizer porque perdemos?%. 
We have tried to develop a system as complete as possible and we created several strategies for each system feature, like exploring, exploiting, buying, repairing, and attacking. Hence we developed many versions of the system, we exhaustively tested each one against the others to select the more efficient. We also tested our system during the contest test phase against the teams provided by the contest organisation. This approach was our main advantage in the contest and one of the reasons we played eighteen matches against six different opponents and won seventeen. However our system has a worse performance when it confronts a passive system because it is not so offensive. If our agents are in a good map zone they do not bother about the opponent: they assume that the opponent is not in a good area. Also, our agents have no focus on defending a conquered zone and this explains the match we lost against Python-DTU during the contest.

Two main strategies were responsible for the good performance of our system: the buying and exploitation strategies. The buying strategy was decisive because it forced our opponents to reinforce their agents spending a lot of their money. In a match against Python-DTU during the tests phase, for example, we conquered a small area but we won because we had more money. Fig.~\ref{hulk1} shows the achievement points from this match. In the step 175 the Python-DTU (in blue) spent most of their money strengthening their agents and SMADAS (in green) spent only a part of its money. In the last 400 steps, from the step 350 to the 750, we had about 23 achievement points in each step, summing 9200 achievement points. In the end, this difference allowed us to win this match, as shown in Fig.~\ref{hulk2}.

%\begin{figure}
%\centering 
%	\includegraphics [width= 12cm] {./AchievementPoints.png}
%	\caption{From the step 350, SMADAS-UFSC (in green) has more achievement points than Python-DTU (in blue) (a). This difference has decided the match for SMADAS-UFSC system (b).}
%\label{fig:achievementpoints}	
%\end{figure}


\begin{figure}
	\centering
	\subfloat[]{\includegraphics[width=0.48\linewidth]{./hulk1.jpg}\label{hulk1}}
	\hspace{0cm} % este comando é para deixar uma distância de 1 cm entre as figuras
	\subfloat[]{\includegraphics[width=0.48\linewidth ]{./hulk2.jpg}\label{hulk2}}
	\caption{From the step 350, SMADAS-UFSC (in green) has more achievement points than Python-DTU (in blue) (a). This difference has decided the match for SMADAS-UFSC system (b).}
	\label{fig:achievementpoints}
\end{figure} 

Our exploitation strategy chooses two good zones in the map. It was efficient because usually the opponents are concerned about finding and conquering just one good zone. Thus while part of our agents are under attack in one of these zones, the other part are scoring in another zone. This strategy earns less points in each step, because our agents are divided in two smaller zones, but it has better results against an offensive opponent.  Fig.~\ref{fig:1x2morrinho} shows a comparison from our system performance using these two exploiting strategies. The system in green tries to conquer one single zone and the blue system looks for two zones. The blue system has fewer points at the beginning because it gets two smaller zones. However after some steps where the green system loses many points disputing a single zone, the blue system has one fixed zone scoring without any attack. This strategy was decisive in the match against the AiWYX system. 

\begin{figure}
	\centering
	\subfloat[]{\includegraphics[width=0.48\linewidth]{./zones1.jpg}\label{zones1}}
	\hspace{0cm} % este comando é para deixar uma distância de 1 cm entre as figuras
	\subfloat[]{\includegraphics[width=0.48\linewidth]{./zones2.jpg}\label{zones2}}
	\caption{The green system tries to conquer one single zone and the blue system looks for two zones. The blue system finishes the match with a highest score because it keeps scoring in a zone without disputing it with opponents.}
	\label{fig:1x2morrinho}
\end{figure} 

% talvez incluir numa nova versao:
% - problema de processar percepcao
% - tempos de processamento, desempenho do time
% - gargalos
% - profilling, e avaliacoes nossoas
% - colocar umas telas do cenario que sejam significativas

  	% Rodrigo
\section{Conclusion}

In our second participation in the MAPC we had again a worthy experience. Our team performed very well and we won the MAPC for the second consecutive year. The strategy to get many small zones was the strongest point of our team and it became more difficult for the opponents to disturb our zones because our agents were spread out over the whole map while our saboteurs were able to disturb the opponent zones. However, our team can be improved to perform better in maps with low thinning (less than 20\%) and with too many good vertices gathered in the same area. The best strategy for it seems to be to conquer a big zone and defend it instead of building small zones.

We also had the opportunity to use new tools, such as the \jacamo\, and to test some issues related to our research topics, such as \emph{count-as rules}. It was a good challenge and we got positive results. The main results were ($i$) the contributions for the improvement of the used tools and ($ii$) the concrete verification that considering the organization and the environment as first-class entities has improved the team program quality.

Finally, as suggestions to improve the current scenario, we suggest that ranged actions be revised to balance the fail probability. So far, it is not a good strategy to use ranged actions, since the agents need to buy several sensors to decrease this probability. 	% Jomi

\bibliographystyle{plain}
\bibliography{bibliografia}

\section*{Short Answers}

\appendix

\section{Introduction}

\begin{enumerate}
\item What was the motivation to participate in the contest?\\
	A: Evaluate the result of our master and PhD thesis.

%Multi-Agent Programming Contest offers an useful context based on cooperation, coordination, and decentralisation to evaluate our master and Phd thesis proposals. Our main approach is ($i$) to develop a \emph{base} multi-agent system for the contest, after it ($ii$) change the base system using our corresponding proposals, and then ($iii$) evaluate and compare  each proposal against the base system. Another motivation is to improve our experience in developing MAS, because most of us are just starting on the domain.\\

\item  What is the (brief) history of the team? (MAS course project,  thesis evaluation, $\ldots$) \\
	A: Our team was formed by members from the Multi-Agent Systems research group (called SMADAS) at Federal University of Santa Catarina (UFSC). %\\

\item  What is the name of your team?\\
	A: Our team's name is SMADAS-UFSC.%\\

\item  How many developers and designers did you have? At what level of education are your team members? \\
	A: Our team has six developers and everyone was involved with the system design. We have one PhD, one PhD student, three masters students and one undergraduate student. %\\

\item  From which field of research do you come from? Which work is related?\\
	A: All team members work with Multi-Agent Systems and Artificial Intelligence. %\\

\end{enumerate}

\section{System Analysis and Design}

\begin{enumerate}
 	\item  Did you use a Multi-Agent programming languages? Please justify your answer.\\
	A: We used the Jason language because all members are familiar with it.%\\ % and it is used in their master and PhD thesis. 
 
 	\item  If some Multi-Agent system methodology such as Prometheus, O-MaSE, or Tropos was used, how did you use it? If you did not, please justify.\\
	A: We did not use any software engineering methodology because the problem seemed quite simple to solve and we had no experience with such methodologies.%\\ %Thus we decided that it was better to use our time developing the system than learning a methodology. 
   	   
 	\item  Is the solution based on the centralisation of coordination/information on a specific agent? Conversely if you plan a decentralised solution, which strategy do you plan to use?\\
	A: The system information is decentralised: each agent has all available information about the enemies and the graph. The coordination is centralised in a few cases, to solve some conflicting situations, like defining which agent should be repaired first or what is the best zone to exploit. %\\
   
	\item  What is the communication strategy and how complex is it?\\
	A: The agents use two mechanisms for communication: a blackboard and message exchanging. Some communication protocols are composed by one single message sent by an agents to others (e.g., when an enemy is inspected or when an agent report its action). Other protocols use more messages, for example when a damaged agent request a repair, nine messages are sent among the damaged agent and the repairers.%\\
		  
    \item  How are the following agent features considered/implemented: \emph{autonomy}, \emph{proactiveness}, \emph{reactiveness}?\\
	A: The agents are autonomous, reactive, and proactive. They have autonomy to decide how and when to execute their actions, they react to environment events and new messages, and are proactive while looking for a better vertex.%\\


	\item  Is the team a truly \textbf{multi}-agent system or rather a centralised system in disguise?\\
	A: The tasks of the team are decentralised among the agents which need to coordinate themselves to produce a coherent global behaviour.%\\
	
	
	\item  How much time (man hours) have you invested (approximately) for implementing your team?\\
	 A: We expended about 500 hours developing the system.%\\


	\item  Did you discuss the design and strategies of you agent team with other developers? To which extend did your test your agents playing with other teams.\\
	A: We did not discuss the design or strategy with other teams before the contest.%\\ % because we tested different strategies in order to decide which would be used in the contest. 
	
\end{enumerate}

\section{Software Architecture}

\begin{enumerate}
	\item  Which programming language did you use to implement the Multi-Agent system?\\
	A: The language used for programming our agents is Jason 1.3.8 \cite{bordini:2007}.%\\
	
	\item  How have you mapped the designed architecture (both Multi-Agent and individual agent architectures) to programming codes, i.e., how did you implement specific agent-oriented concepts and designed artifacts using the programming language?\\
	A: The BDI concepts provided by the Jason language are the building blocks to develop our strategies.%\\ %It allows the creation of plans so the agents can complete their goals and the implementation of more than only one strategy, each one described by a plan. Our agent communication has two ways: agent to environment and agent to agent. In the first situation in each step the framework EISMASSim receives a XML file with the percepts of the agents. The agent-to-agent communication uses the Jason's performative communication.
	
	\item  Which development platforms and tools are used? How much time did you invest in learning those?\\
	A: We used Eclipse platform with Jason 1.3.8 plug-in. These tools were known by all team members then we spend just few hours learning new features. %\\
	
	\item  Which runtime platforms and tools (e.g. Jade, AgentScape, simply Java, $\ldots$) are used? How much time did you invest in learning those?\\
	A: We used EISMASSim framework to communicate with the environment and spent about 50 hours to learn it. For communication among the agents, we used Jason centralised infrastructure.%\\
		
	\item  What features were missing in your language choice that would have facilitated your development task?\\
	A: The Jason language has almost all features we needed to program our agents. However, for some algorithms, we preferred Java because it is faster.%\\
	
	\item  Which algorithms are used/implemented?\\
	A: We used two traditional algorithms for graphs: Dijkstra and breadth-first search.%\\ %algorithm to find the best path between vertices. The second is the breadth-first search algorithm to locate the best area in the graph. These algorithms were implemented using Java methods and our BDI agents used internal actions to invoke these methods.
	
	\item  How did you distribute the agents on several machines? And if you did not please justify why.\\
	A: The agents were conceived to execute in the same machine to simplify blackboard programming, which uses shared memory. Future versions of the system will use distributed blackboards.%\\
		
	\item  To which extend is the reasoning of your agents synchronized with the receive-percepts/send-action cycle?\\
	A: The synchronisation with the environment is given by the reasoning cycle of Jason, where the first step includes the perception and the last the action.%\\
		
	\item  What part of the development was most difficult/complex? What kind of problems have you found and how are they solved?\\
	A: A blackboard has been used to share and build the knowledge about the environment. The process to update information in the graph has a high computational cost, lasting more than one step. Therefore, to avoid losing steps, the graph is updated and shared every three steps.%\\
		
	\item  How many lines of code did you write for your software?\\
	A: We have 7885 lines of code, 5504 written in Jason and 2381 written in Java.%\\
		
		
\end{enumerate}

\section{Strategies, Details and Statistics}

\begin{enumerate}
	\item  What is the main strategy of your team?\\
	A: We conceived our system strategy in two main phases: exploration, in which the explorers identify all vertices and nodes in the map and the best zones, and exploitation, where all agents try to conquest and defend these zones.%\\
	 
 	\item  How does the overall team work together? (coordination, information sharing, ...)\\
	A: Our agents exchange information to coordinate their activities.%\\ %This information exchange also helps to prevent redundant actions.

	\item  How do your agents analyse the topology of the map? And how do they exploit their findings?\\
	A: Some important information about the graph structure is shared and synchronized in the blackboard and it is used by the agents to move through the map. Despite this, agents do not use any information about topology to make the decisions. %\\

	\item  How do your agents communicate with the server? \\
	A: We use the EISMASSim framework to communicate with the server. External actions and usual perception are used by the agents to interact with the EISMASSim.%\\

	
	\item  How do you implement the roles of the agents? Which strategies do the different roles implement?\\
	A: The implemented strategies for each agent type is shown in Table~\ref{tab:tabStrategies}.%\\
	
	\item  How do you find good zones? How do you estimate the value of zones?\\
	A: The system uses a modified version of the BFS algorithm to find the best zones in the map. It is run for all vertices, summing their values until some depth. The vertex with the highest sum represents where the best zone is (zone 1). After it, the algorithm tries to find the second best vertex to set the second best zone (zone 2).%\\ %This algorithm is not optimal because its result is always a circular shape, when often the ideal choice is a free shape.


	\item  How do you conquer zones? How do you defend zones if attacked? Do you attack zones?\\
	A: With the zones defined, each agent is informed about the central vertex of its zone and how far they can travel inside it. The distance they can travel is the shortest path, in  number of edges, between the central vertex and the target vertex. To defend these zones, the saboteurs attack all opponents inside the zone or in nearby vertices. The other agents stay in a vertex that has two neighbour vertices that belongs to our system. It is assumed that if the enemy zone is not near, the opponent likely has a small zone and then our agents do not try to attack it.%\\


	\item  Can your agents change their behaviour during runtime? If so, what triggers the changes?\\
	A: If the opponent does not have any buying strategy, the \texttt{Hulk} agent changes its behaviour and it stops buying upgrades. Besides it, in the start of the match the saboteurs attack the enemies, but after some steps they change their behaviour to attack the enemies. %\\


	\item  What algorithm(s) do you use for agent path planning? \\
	A: We used Dijkstra to path planning.%\\


	\item  How do you make use of the buying-mechanism?\\
	A: It was defined the minimum that the agents have to buy in order to make the enemy expend its money. In particular, we have \emph{one} agent (named \texttt{Hulk}) that focus on buying and inducing all the opponents to also buy and spend their money.%\\ %One agent (named \texttt{Coach}) was chosen to receive information about the opponents and if necessary this agent informs to \texttt{Hulk} to stop buying. 


	\item  How important are achievements for your overall strategy?\\
	A: The achievement points are quite important since they accumulate each step. It is desirable to get the maximum of achievement points as soon as possible, but some achievements are hard to get. For example, our system does not surveys all edges and they do not inspect all opponents because it takes a long time and it is better to keep the agents in the best vertices, getting water wells score. %\\


    \item  Do your agents have an explicit mental state?\\
	A: The agents have their beliefs and use them to reason about their next action.%\\ %The agents also keep in their beliefs base many information to exchange with other agents.


	\item  How do your agents communicate? And what do they communicate?\\
	A: Our agents communicate indirectly by using the blackboard and directly by message exchanging. %\\ %In the blackboard all information about the graph structure is synchronized. By the message exchanging the agents transmit informations about the enemies, friends, executed actions, damages, map zones, vertices and edges. 
	
	\item  How do you organize your agents? Do you use e.g. hierarchies? Is your organization implicit or explicit?\\
	A: There is an explicit pre-defined hierarchy to prevent redundant actions: agents with higher priority decide before the others.%\\


	\item  Is most of your agents’ behavior emergent on an individual or team level?\\
	A: In our strategy both individual and group behaviour are important. The individual behaviour is important when the agents are isolated in the map trying to get achievement points. The group behaviour is responsible for preventing redundant actions and conquering zones, for example.%\\


	\item  If your agents perform some planning, how many steps do they plan ahead?\\
	A: We do not use planning, all plans are previously programmed based on the strategies. %\\
          %Our agents do not plan a lot of steps ahead because the environment changes fast, and so the plans must be adapted to it. In most cases the agents react in just one step. Some situations when they plan with some steps ahead is when there is a disabled agent or unprobed vertex. 

	\item  If you have a perceive-think-act cycle, how is it synchronized with the server? \\
	A: We use the EISMAssim framework \cite{behrens:2011} to synchronize the agent actions to the server.%\\
	
\end{enumerate}

\section{Conclusion}
\begin{enumerate}
	\item  What have you learned from the participation in the contest?\\
	A: We learned a lot about MAS developing and about the tools and languages we used. %\\
		
	\item  Which are the strong and weak points of the team?\\
	A: Our strongest point is that we created several strategies for each system feature and  tested them against each other to select the more efficient ones. Our weakness is that our system is not so offensive. Another problem is that our agents does not focus on defending their own zone.%\\
		
	\item  How suitable was the chosen programming language, methodology, tools, and algorithms?\\
	A: The Jason programming language was quite mature and suitable for the agent programming. However, we still need tools for programming and debugging at a higher level of abstraction.%\\ %supports agent programming with abstract concepts like plans, beliefs, and goals which are suitable for the problem and very expressive. We did not identify any bug on Jason which shows the maturity of this kind of language. Although we can evaluate the used tools positively in general, some features are still missing. For example, it was very difficult to change, refactor, and debug the agents code. The tools provided by Jason for debugging, like the sniffer and the mind inspector, are too specific and focused on the details.

	\item  What can be improved in the contest for next year?\\
	A: We can improve our system both in the strategies and the tools. Our system is focused only on the agent aspect and more global aspects should be considered.%\\ %, for instance, for organisation and interaction programming as first class abstractions.
	
	\item  Why did your team perform as it did? Why did the other teams perform better/worse than you did?\\
	A: Our system performed well because we focused on extensively testing all strategies. %\\
	
	\item  Which other research fields might be interested in the Multi-Agent Programming Contest?\\
	A: We think that some parts of the problem can be solved by optimisation techniques, which we plan to use in future versions of the systems.%\\
	
	\item  How can the current scenario be optimized? How would those optimizations pay off? \\
	A: We propose two improvements. ($i$) Inform opponent's score. It would allow participants to design strategies based on the current match result, rising more confrontations. ($ii$) Leave the graph less connected to increase the use of edges.%\\ %Currently the edges do not influence the system strategies but a less connected graph forces the systems to work with the edges weight.
	
	
\end{enumerate} 	% Dani
%---------------------------------

\end{document}
