\section{Conclusion}

In our second participation in the MAPC we had again a worthy experience. Our team performed very well and we won the MAPC for the second consecutive year. The strategy to get many small zones was the strongest point of our team and it became more difficult for the opponents to disturb our zones because our agents were spread out over the whole map while our saboteurs were able to disturb the opponent zones. However, our team can be improved to perform better in maps with low thinning (less than 20\%) and with too many good vertices gathered in the same area. The best strategy for it seems to be to conquer a big zone and defend it instead of building small zones.

We also had the opportunity to use new tools, such as the \jacamo\, and to test some issues related to our research topics, such as \emph{count-as rules}. It was a good challenge and we got positive results. The main results were ($i$) the contributions for the improvement of the used tools and ($ii$) the concrete verification that considering the organization and the environment as first-class entities has improved the team program quality.

Finally, as suggestions to improve the current scenario, we suggest that ranged actions be revised to balance the fail probability. So far, it is not a good strategy to use ranged actions, since the agents need to buy several sensors to decrease this probability.