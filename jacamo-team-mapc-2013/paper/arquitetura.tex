\section{Software Architecture}

As we done in the edition of 2012~\cite{smadas:2012}, in the current edition of MAPC we used the EISMASSim~\cite{behrens:2011} to communicate with the contest server. However, while in the previous edition the team was developed essentially using Jason, in the current edition, our team has been developed with \jacamo\  platform. This was the main change in the software architecture for this year. Furthermore, even with the increase number of agents, from 20 to 28, we were still able to run the agents in a single machine, therefore we decided to avoid distributing the agents between several machines. We also made several contributions for the tools that we used in this year. In Jason, we added features to handle goals with deadlines, new mechanisms for the \texttt{.wait} internal action, and we fixed some bugs. In \moise, we added a new feature to reset organizational goals to avoid creating new schemes at runtime and we added an organizational monitor accessed via HTTP, so that we were able to watch our team organization remotely.

The source code of the team has 3794 lines of Jason code, 135 for \moise, 96 for \cartago, and 4434 for Java, totaling 8459 lines. Although the implemented strategies of these year are more complex, we can notice that the number of lines coded in Jason has decreased from 5504 in last year's team to 3794 this year. It is an expected consequence of the organization and the environment programming available in \jacamo. Coordination strategies that previously required several lines of Jason code, could now be coded in a few lines of \moise, since \moise\ is a proper language for that.  Not only have we reduced the size of the programs, but the new approach has allowed us to debug and change the organization of the team quite easily. Instead of monitoring the agents internal state, we can now monitor the state of the organization, which is a more general view of the state of the team. Since the organizational program is the same as the specification, to change the team sometimes is simply reduced to update the organization. For instance, to change the order of organizational goals, we simply need to change the scheme of Fig.~\ref{fig:org_fs}.