% +- 1 pagina
\section{Introduction}

The Multi-Agent Programming Contest (MAPC)~\cite{koster:2013}\footnote{\url{http://multiagentcontest.org}} is an important event to stimulate research in the multi-agent systems programming field. The MAPC 2013 used the ``Agents on Mars'' scenario, which was improved from the last year scenario, therefore the efforts must continue concentrated in cooperation, coordination, and decentralization. Our agent team, called SMADAS, acronym for our research group, named Multi-Agent Systems from Systems and Automation Department (in Portuguese, \textbf{S}istemas \textbf{M}ulti\textbf{a}gentes  do \textbf{D}epartamento de \textbf{A}utoma\c{c}\~ao e \textbf{S}istemas) was developed by a group formed by one PhD, four PhD students, and two undergraduate students from the Federal University of Santa Catarina (UFSC). This is our second participation in the contest and we have two main aims this year: improve our MAS developing skills and evaluate some proposals developed in our thesis. 

